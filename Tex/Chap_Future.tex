\chapter{总结与展望}\label{chap:future}

\section{总结}
本文是基于深度学习的基础上,使用从天津肿瘤医院采集的真实数据进行训练,通过对医学数据,尤其是乳腺钼靶的数据的深入分析,结合医院的真实需求,确定评判标准,对数据进行数据预处理以满足传统计算机视觉所要求的数据格式,之后进行数据集划分。考虑到数据集中带有标注的恶性病例数据以及未带标注的正常病例数据,初步设想是设计基于Faster R-CNN的先训练恶性病例数据再训练学习恶性病例与正常病例数据之间区别的两阶段模型weakFaster R-CNN,但由于在训练过程中发现此模型存在耗时长,占用空间等诸多问题,于是创新性地在传统的检测模型Faster R-CNN基础上,提出simFaster R-CNN模型,针对数据特点分别在RPN模块设计针对正常样本的top likelihood loss及在Fast R-CNN部分针对学习正常与恶性之间区别的similarity loss,在保证一定精度的前提下,有效地加快了模型训练和测试的速度。

本文是基于广泛存在于医疗影像数据中存在的带有标注的恶性数据及未带标注的正常数据的分析上,在提供一般思路的前提下,提出全新问题,设计全新的损失函数,来为后期的这种数据类型提供开拓性的思路。 
具体创新点和主要贡献如下:
\begin{itemize}
	\item 首次考虑医学数据中常见的未全标注信息的弱监督数据; 
	\item 提出适用于医院实际需求的评判标准;
	\item 使用weakFaster R-CNN分阶段学习弱监督数据,完成基于深度学习的乳腺钼靶分类检测工作; 
	\item 提出simFaster R-CNN,引入度量学习,设计全新损失函数,在学习恶性肿瘤的同时,能够同时区分学习恶性和正常乳腺钼靶之间的区别,满足时间和性能的要求。
\end{itemize}

\section{展望}
当今世界,以人工智能技术为主要代表的第四次工业革命已经风起云涌,人工智能技术已经逐步走向成熟,各大公司也开始纷纷将技术与实际领域相结合,以交叉学科的方式将技术应用于产品,从而将产品落地,更好地服务于社会,以及实现公司盈利的前景。在其中,医疗领域无疑以数据量大,数据格式丰富,与国计民生紧紧联系在一起成为各个公司纷纷布局的市场,其中以国家发布的人工智能发展规划中提到的腾讯觅影计划最为著名。通过第一次完完全全从合同签订,到数据采集,标准制定,数据处理,模型设计与分析,模型优化等整个流程走完,我对真实场景的业务数据有了更为深入的了解,并对医疗影像数据有了更为透彻的把握,对于乳腺钼靶图片数据是否患癌也能做个大致的判断。

真实医学场景复杂,数据格式更为多变,每个医院的标准也并不相同,在医学领域的人工智能应用还有漫长的道路需要前进,短短的基于深度学习的乳腺钼靶分类算法只能说是所作的工作的冰山一角。未来肯定会涌现出越来越多的基于深度学习的基于人工智能技术发展起来的医疗影像,乃至医学数据的算法与应用。唯有当技术更好地与领域相结合,技术才能更好地造福于人类,才能更好的服务于社会。

感谢本项目,让我独自负责并全程跟踪一个项目,从而有效锻炼了我对数据的嗅觉以及编码能力。本文还需要进一步研究和改进的是:
\begin{itemize}
	\item 针对良性和阴性数据之间的更加细微的区别应该单独考虑,所以有必要更加细化top likelihood loss和similarity loss。
	\item 模型的输入目前是一正一负,在这方面可以考虑引入同正,同负等情况,这样可以增加数据的多样性,一定程度上对数据进行了扩增,从而让模型训练更加充分。
	\item 目前针对这种弱监督学习的情况,传统的计算机视觉还有最新研究,比如Domain Adaptation,异常检测\cite{chandola2009anomaly}等方法,可以适时地引入到乳腺钼靶的分类检测当中。
	\item 一个卓越的主干网络可以有效地提高模型的整体预测水平,可以通过引入全新的更加高效的主干网络来提升模型的整体性能。
	\item 乳腺钼靶本身的dicom格式数据还包含诸多信息,比如病人的年龄,性别等,可以考虑使用结构化数据设计机器学习模型,通过基于深度学习的乳腺钼靶分类检测模型与机器学习模型进行融合,增强模型分类的鲁棒性。
	\item 乳腺钼靶还存在钙化,特征征象,结构扭曲等病灶类型。钙化最明显的特征是明亮,数量较多,但难度在于分布各异,大小仅为0.5mm到1mm,这给检测提出了全新的难度,原有的simFaster R-CNN需要着重修改RPN模块,提高细小物体的识别能力;对于特征征象,其关键特征在于一点发出放射状,有局灶性收缩或实质的边缘扭曲等现象,其本身并没有良恶之分,可以直接使用传统的Faster R-CNN模型训练足矣;对于结构扭曲,其需要通过两个乳房综合判断,所以在模型设计时,需要将单个乳房两个乳房同时输入,判断二者乳房的差异性,进而得出结果。
	比如钙化病灶较为明亮,
	\item 由于在医学数据上存在大量的正常样本中存在恶性病灶信息,且正常样本无标注的情况,比如肺结节病灶识别,胃癌病灶识别等等,simFaster R-CNN模型需要引入更多的数据量以及更多不同的病种数据,对模型进行跨数据的测试,从而让模型能够成为通用的解决这类数据问题的范本。
\end{itemize}


