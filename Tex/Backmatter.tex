
\begin{comment}
\chapter[致谢]{致\quad 谢}\chaptermark{致\quad 谢}% syntax: \chapter[目录]{标题}\chaptermark{页眉}
\thispagestyle{noheaderstyle}% 如果需要移除当前页的页眉
%\pagestyle{noheaderstyle}% 如果需要移除整章的页眉

光阴似箭,转眼之间,研究生三年已然结束,回首往事,这三年的成长故事历历在目。从研一时对科研生活的懵懂,到研二研三在项目,比赛,工作和发论文中不断完善个人能力,成功地从一名web开发工程师转变为一名算法工程师,也从对深度学习,人工智能的一无所知到而今的小有所成。一路走来,除了和时间赛跑,在其中也经历了很多挫折,有第一次跟进深度学习便独自承担医院的项目与甲方合作交流的稚嫩,有为了找工作突出能力多条线打比赛的突进,有健身身体受伤和长时间失眠等等。这一路走来,汗水与成果并重,最后取得的成果也达到了个人的预期设想。感谢母校,计算所为我提供了良好的学习环境,使我能够在此专心学习。

首先,我要感谢我的导师赵屹研究员。在这几年的研究学习过程中,每一次和赵老师的交流都让我获益匪浅,他关于科研和项目的见解让我更加地深刻地了解研究生以及未来的岁月中应该做什么。赵老师为人处世上更是值得我去学习,总之,我十分庆幸自己能够成为赵老师的学生并接受他的悉心指导。

其次,我还要感谢我的指导教师肖立老师。肖立老师带着我去接洽项目,指导模型的设计与实践,在项目中指导我如何进行思考,如何进行论文的撰写,有力地推动了个人项目的进展。在这过程中,相比于在项目中个人的困难,个人的能力成长更多,也学会了如何更好地与别人交流,如何更好地发挥个人的优势,尽己之力为所在团队做出个人的贡献。

另外,我还要感谢在实验室一起工作,一起成长的师兄罗纯龙,师弟师妹罗宇凡,刘阳,乔艺璇等,在科研和项目过程中和他们一起交流沟通,一起嘻嘻哈哈,研究生生活才有了更多的生活味道。同时也十分感谢计控3班的同学们,不论是在雁栖湖集中教学的时光还是在计算所平时碰面时的点头微笑,我们在学业上互帮互助,在生活中一起释放青春活力。值此毕业之际,感谢你们的陪伴,也祝你们前程似锦,一帆风顺。

最后要感谢的是我的父母,儿行千里母担忧,父母对我的殷殷期盼成为我努力奋斗,积极向上的目标。感谢父母在我研究生生涯中对我工作和学习的支持,父母,你们辛苦了!

在这里,我衷心地感谢所有在我成长路途中帮助过我的人,因为有你们的陪伴,我的研究生才如此多姿多彩!祝你们万事如意,心想事成!
\end{comment}
\chapter{作者简历及攻读学位期间发表的学术论文与研究成果}

\section*{作者简历}

2012年9月——2016年6月,在西北工业大学软件与微电子学院获得学士学位。


2016年9月——2019年6月,在中国科学院计算技术研究所攻读硕士学位。

\section*{已发表 (或正式接受) 的学术论文:}
Learning from Suspected Target: Bootstrapping Performance for Breast Cancer Detection in Mammography(MICCAI会议在投)

\section*{参加的研究项目及获奖情况:}

\subsection{参加的研究项目}
2017年10月——2018年8月,天津肿瘤医院乳腺钼靶分类项目

2018年1月——2018年2月,阿里天池糖尿病预测比赛

2018年8月——2018年9月,深圳医疗健康大数据创新应用国际大赛

2018年9月——2018年11月,宁波二院胃镜及皮肤镜项目

2018年11月——至今,宁波二院肺结节项目

\subsection{获奖情况}
2016年——2017年国科大三好学生

阿里天池fashionAI服饰属性识别大赛亚军

第四届腾讯广告算法大赛第四名




\cleardoublepage[plain]% 让文档总是结束于偶数页,可根据需要设定页眉页脚样式,如 [noheaderstyle]

