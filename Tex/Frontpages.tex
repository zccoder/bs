%---------------------------------------------------------------------------%
%->> 封面信息及生成
%---------------------------------------------------------------------------%
%-
%-> 中文封面信息
%-
\confidential{}% 密级:只有涉密论文才填写
\schoollogo{scale=0.095}{ucas_logo}% 校徽
\title{基于深度学习的乳腺钼靶分类算法研究}% 论文中文题目
\author{朱城}% 论文作者
\advisor{赵屹~研究员

中国科学院计算技术研究所}% 指导教师:姓名 专业技术职务 工作单位
\advisors{}% 指导老师附加信息 或 第二指导老师信息
\degree{硕士}% 学位:学士、硕士、博士
\degreetype{工学}% 学位类别:理学、工学、工程、医学等
\major{计算机应用技术}% 二级学科专业名称
\institute{中国科学院计算技术研究所}% 院系名称
\date{2019~年~6~月}% 毕业日期:夏季为6月、冬季为12月
%-
%-> 英文封面信息
%-
\TITLE{Research on Classification Algorithm of Mammography Based on Deep Learning}% 论文英文题目
\AUTHOR{Zhu Cheng}% 论文作者
\ADVISOR{Supervisor: Professor Zhao Yi}% 指导教师
\DEGREE{Master}% 学位:Bachelor, Master, Doctor。封面格式将根据英文学位名称自动切换,请确保拼写准确无误
\DEGREETYPE{Science in Engineering}% 学位类别:Philosophy, Natural Science, Engineering, Economics, Agriculture 等
\MAJOR{Computer Application Technology}% 二级学科专业名称
\INSTITUTE{Institute of Computing Technology, Chinese Academy of Sciences}% 院系名称
\DATE{June, 2019}% 毕业日期:夏季为June、冬季为December
%-
%-> 生成封面
%-
\maketitle% 生成中文封面
\MAKETITLE% 生成英文封面
%-
%-> 作者声明
%-
% \makedeclaration% 生成声明页
%-
%-> 中文摘要
%-
\intobmk\chapter*{摘\quad 要}% 显示在书签但不显示在目录
\setcounter{page}{1}% 开始页码
\pagenumbering{Roman}% 页码符号

乳腺癌是女性最常见的恶性肿瘤之一,传统的基于乳腺钼靶的CAD系统在乳腺癌的分类诊断上存在耗时长、较高人为误诊率等问题。目前基于深度学习的乳腺钼靶所使用的整图分类及病灶检测等方法并没有综合考虑医学数据中存在数据部分标注的情况。为此本文主要进行了深度学习技术在乳腺钼靶这种不完全数据下的算法研究。

乳腺钼靶这类医疗影像分类问题有别于传统自然图像,其分辨率大、数据采集困难、数据量小、数据未全标注、类别之间差别微小等问题对研究者们提出了更高的要求和挑战。因此,本文先对医疗影像进行了分析,并在此基础上设计满足医生实际需求和计算机视觉领域要求的评判标准,对数据进行预处理等;为了充分利用医疗影像存在的未标注全的正常样本数据,本文在深度学习物体检测模型Faster R-CNN基础之上,引入弱监督学习的方法,在对带有标注信息的恶性乳腺钼靶病例充分学习之上,创新性地利用未带标注信息的正常样本数据,分别设计了两阶段模型weakFaster R-CNN和一阶段检测模型simFaster R-CNN。

weakFaster R-CNN采用两阶段训练方式,第一阶段专门学习恶性样本信息,第二阶段学习恶性及正常样本之间的区别。但weakFaster R-CNN存在耗时长、占用空间大等问题,为此针对数据特点提出simFaster R-CNN。
simFaster R-CNN通过引入度量学习,创新性地提出了top likelihood loss和similarity loss等两个损失函数分别优化RPN网络的正常样本和学习正负样本之间的微小差异,取得了不逊于weakFaster R-CNN的效果。本文并使用该方法在大型公开数据集上DDSM上进行了测试,可以媲美目前基于全标注信息的模型训练的结果,充分证明了一阶段模型的可行性和优越性。而且该模型对于医学数据具有普适性,后期可进行更加深入的研究,具有充分的实际应用价值。


\keywords{乳腺钼靶,weakFaster R-CNN,simFaster R-CNN,度量学习,similarity loss}% 中文关键词
%-
%-> 英文摘要
%-
\intobmk\chapter*{Abstract}% 显示在书签但不显示在目录


Breast cancer is one of the most common malignant tumors in women. The traditional CAD system for mammography has lots of problems, including a long time-consuming diagnosis of breast cancer and a high rate of misdiagnosis. 
And the current methods for mammography based on deep learning which applied for the classification of whole images and the detection of lesions do not comprehensively consider the data without annotations in the medical area. 
So this paper mainly studies the algorithm of deep learning technology under the circumstances of such incomplete data of mammography.

The image classification problem of mammography is different from that of traditional natural images,which includes the large resolution, the difficulty of data collection, the small datasets, and the not fully labelled of training data. 
The problem above have put forward higher requirements and challenges to the researchers. So this paper firstly analysed the medical images, formulated the relevant evaluation criteria which took care of doctor's and computer vision's need and preprocessed the data after that.
In order to make full use of the unlabelled normal data of medical images, this paper used the deep learning object detection model Faster R-CNN and introduced weak supervised learning.
After trained enough target cases of the malignant mammography with labelled information, this paper innovatively made use of normal data without labelled information.And then, this paper proposed a two-stage detection model(weakFaster R-CNN)and a one-stage detection model(simFaster R-CNN)respectively.

weakFaster R-CNN used a two-stage training approach, in which the first stage focused on studying malignant information and the second stage learned the difference between malignant and normal images. Result from the problems of weakFaster R-CNN which took a long time and took up a lot of space, simFaster R-CNN was proposed after then.
simFaster R-CNN introduced metric learning, including top likelihood loss and similarity loss, to optimize the loss of normal images in the RPN and the tiny difference between the positive and negative images respectively, finally achieved the effect which was as good as that of weakFaster R-CNN. This method was also used to be trained at the large public datasets DDSM.Compared to the the results of model trained at full annotated information, simFaster R-CNN even can get a better result.
Last but not least, the model has universality for medical data, and can be further studied in the future, which makes it have full practical application value.

\KEYWORDS{Mammography, weakFaster R-CNN, simFaster R-CNN, Metric Learning, similarity loss}% 英文关键词
%---------------------------------------------------------------------------%
